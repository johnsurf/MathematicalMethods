\section{\underline{Thermodynamics}}

\begin{definition}Path Dependent Differentials: A variable $X$ whose variation from equilibrium state $A$ to equilibrium state $B$, depends on the path has its derivative notated by $\dbar X$. \end{definition}

Example of path dependent thermodynamic quantities: \[ \mbox{1) Heat } Q \quad \quad \mbox{ 2) Work } W. \] 

\begin{definition} State Variable: If the integral of a quantity around any closed path is zero, that quantity is called a state variable, that is it has a value that is characteristic only of the state of the system, regardless of how that state
was arrived at. \end{definition}

{\bf Zeroth Law of Thermodynamics}: If $A$ and $B$ are in thermal equilibrium with a third body $C$ (the ``thermometer''), then $A$ and $B$ are in thermal equilibrium with each other. 

{\bf First Law of Thermodynamics}: Every thermodynamic system in an equilibrium state possesses a state variable called the Internal Energy, $U$, (a single-valued function) whose change $dU$ in a differential process is given by
Eq. (\ref{1stLaw}). 
\[ dU = \dbar Q - \dbar W \label{1stLaw}\]

{\bf Second Law of Thermodynamics}:\\
1) Clausius: It is impossible for any cyclic machine to produce no other effect than to convey heat continuously from one body to another at a higher temperature \\
2) Kelvin and Planck: A transformation whose only final result is to transform into work heat extracted from a source which is at the same temperature throughout is impossible.
\begin{definition} Every thermodynamics system in an equilibrium state possesses a state variable called the Entropy, $S$, whose change in a differential process is given by
\[ dS = {\dbar Q\over T}\quad \quad\quad\oint \, dS = 0 \] \end{definition}


{\bf Alternate form of the Second Law of Thermodynamics}:\\
3a) A natural process (reversible) that starts in one equilibrium state and ends in another can go equally well in either direction, and for reversible processes the entropy of the system plus environment remains unchanged. \\
3b) A natural process (irreversible) that starts in one equilibrium state and ends in another will go in the direction that causes the entropy of the system plus the environment to increase. 

\subsection{\underline{Stefan Bolzmann Law}}

In the case of a gas one can relate the pressure and the change in volume to the work done by  the process \[ \dbar W = -p dV. \] Also we can write the change in heat to the temperature and entropy \[ \dbar Q = T dS. \]

In this case the First Law of Thermodynamics is \[ dU = T dS - p dV. \]